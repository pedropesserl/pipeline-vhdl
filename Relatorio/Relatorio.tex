\documentclass[a4paper, 11pt]{article}
\usepackage[top=3cm, bottom=3cm, left = 2cm, right = 2cm]{geometry}
\usepackage[brazilian]{babel}
\usepackage{setspace}
\usepackage{graphicx}
\usepackage{multicol}

\title{Relatório: Trabalho Final - MIPS Pipeline}
\author{Luize Cunha Duarte - GRR20221232\\
    Gabriel Lisboa Conegero - GRR20221255\\
    Pedro Folloni Pesserl - GRR20220072\\
    Rebeca Soares de Oliveira - GRR20221260\\
\textit{Departamento de Informática}\\
\textit{Universidade Federal do Paraná - UFPR}\\
Curitiba, Brasil\\
\texttt{lcd22@inf.ufpr.br, glc22@inf.ufpr.br,}\\
\texttt{pfp22@inf.ufpr.br, rso22@inf.ufpr.br}}
\date{}

\begin{document}
\maketitle

\begin{abstract}
\begin{singlespace}
Este relatório documenta o processo de implementação do processador MIPS Pipeline de 16
bits para FPGA (Field Programmable Gate Array) através da linguagem VHDL (VHSIC Hardware
Description Language). O projeto do processador possui uma memória de dados e uma
memória de instruções de 128 KiB cada, e um banco de registradores com quatro
registradores. 
\end{singlespace}
\end{abstract}

\section{As instruções.}
As divisões dos bits para os endereços e funcionalidades das instruções foi pensada
visando evitar bits soltos e manter um espaço considerável para cada funcionalidade.
Por isso, as instruções seguem a base do MIPS, tendo sua divisão reorganizada para
16 bits.
\begin{itemize}
    \item Tipo R (opcode: 3 bits; rs: 2 bits; rt: 2 bits; rd: 2 bits; shamt: 4 bits;
        func: 3 bits).
        \begin{figure}[h]
        \centering
        \includegraphics[width=.6\textwidth]{tipo-r.png}
        \caption{Formato das instruções do tipo R.}
        \end{figure}

    \item Tipo I. (opcode: 3 bits; rs: 2 bits; rt: 2 bits; imediato: 9 bits).
        \begin{figure}[h]
        \centering
        \includegraphics[width=.6\textwidth]{tipo-i.png}
        \caption{Formato das instruções do tipo I.}
        \end{figure}
\end{itemize}
As instruções suportadas são:
\begin{multicols}{4}
\begin{itemize}
        \item ADD
        \item AND
        \item OR 
        \item NOR
        \columnbreak
        \item Set Less Than
        \item Shift Left Logical
        \item Shift Right Logical
        \item ADDI
        \columnbreak
        \item ANDI
        \item Display
        \item ORI
        \columnbreak
        \item Branch On Equal
        \item Load Word
        \item Store Word
\end{itemize}
\end{multicols}

\section{A arquitetura e sua implementação.}
Buscando uma arquitetura que suportasse todas as instruções decididas, sua organização
deu-se da maneira representada abaixo. Nesse sentido, cada componente foi implementado
utilizando a linguagem VHDL.

\begin{figure}[h!]
    \centering
    \includegraphics[width=.9\linewidth]{arquitetura.png}
    \caption{Bloco operacional do MIPS Pipeline.}
\end{figure}

\section{Os estágios e componentes.}
\begin{enumerate}
    \item \textbf{Instruction Fetch}.

    Sendo a primeira parte da execução de uma instrução, é onde o ponteiro para a
    memória de instruções (Instruction Memory) é calculado e ela é lida, guardando
    tal instrução e sua localização no registrador intermediário IF\_ID. O PC pode
    ser atualizado para a próxima instrução, somando um*, ou para o endereço do Branch.

    *OBS: Em VHDL, a memória de instruções foi implementada como uma array de vetores
    de 16 bits; portanto, ela é indexada pela palavra e não pelo byte. Dessa forma,
    a próxima posição do vetor é simplesmente a posição atual adicionada de 1.

    \item \textbf{Instruction Decode}.

    Após lida, a instrução será decodificada, gerando seus sinais de controle através
    da Control Unit, a qual recebe os três primeiros bits e retorna os sinais necessários
    ligados. Além disso, os registradores usados para realizar a operação desejada são
    lidos no Register Bank, que possui quatro registradores de 16 bits de capacidade,
    sendo o primeiro sempre 0. Por fim, tem-se também nesse estágio a Hazarding Unit,
    a qual reconhece se a instrução sendo executada na proxima etapa irá desviar o fluxo
    de instruções, zerando as instruções que não serão mais executadas, um processo
    denominado “flush”. No flush, os sinais de controle nos registradores IF\_ID,
    ID\_EX e EX\_MEM são zerados, efetivamente descartando as três instruções anteriores
    ao branch que causou um desvio no código.

    \item \textbf{Execute}.

    Tendo os dados necessários para realizar a operação, nessa etapa temos a ULA,
    controlada pela ALUControl, a qual verifica se é uma instrução do tipo R para
    realizar o “func” –- últimos três bits da instrução –- ou então uma soma,
    subtração, AND ou OR.

    Nessa parte, também é calculado o endereço do Branch On Equal, bastando somar ao
    novo PC o imediato estendido com sinal, passado pelos registradores intermediários.
    Por fim, um componente importante para o desempenho do pipeline, a Forwarding Unit,
    é implementado nessa etapa. Esse componente é responsável por evitar stalls no
    processador ao receber dados de etapas seguintes, não sendo necessária a espera da
    escrita nos registradores –- detalhada mais à frente –- das instruções anteriores.

    \item \textbf{Memory}.

    O único componente presente é a memória de dados (Memory Data), o componente mais
    lento do Pipeline. Usada apenas nas funções de Load Word e Store Word, a memória
    possui 512 Bytes de armazenamento com 16 bits cada linha. Por fim, os dados carregados
    são passados para o registrador intermediário MEM\_WB.
    
    \item \textbf{Write-Back}.

    Sendo a última etapa do MIPS Pipeline, ela guarda no banco de registradores o
    dado de saída da ALU ou a palavra carregada da Memory Data. Também é nesse estágio
    que o display acessa o registrador que foi determinado na instrução \texttt{dsp}
    e decodifica seu conteúdo para ligar os segmentos necessários no componente.
\end{enumerate}

\section{O clock.}
    O Pipeline opera a uma frequência diferente do FPGA, tendo um clock mais lento
    que facilite a visualização de seu funcionamento. Para garantir isso, foi criado
    um componente para calcular o novo clock que será utilizado por todo o circuito ---
    o divisor de clock.

\section{O programa de teste.}
    Foi escrito um programa simples como teste das funcionalidades do MIPS Pipeline. O 
    programa calcula os primeiros números da sequência de Fibonacci, e a lógica interna
    cria sinais de saída para displays de sete segmentos, para cada dígito dos resultados
    (que são armazenados em registradores no processador).

\section{Como compilar e carregar o software no FPGA.}
    O MIPS foi desenvolvido usando a plataforma Xilinx ISE, e seu código foi compilado
    para o FPGA Digilent Nexys 3. Para carregar o código na placa, é necessário primeiro
    baixar a plataforma e compilá-lo. As instruções para instalação do Xilinx ISE e do
    software Digilent Adept (usado para carregar o código) estão descritas no arquivo
    \texttt{install.md}, entregue no pacote \texttt{pipeline-vhdl.tar.gz}.

    Após instalados, deve-se abrir um novo projeto na ISE, com as configurações do
    FPGA utilizado (no caso, \textit{Family}: Spartan6, \textit{Device}: XC6SLX16,
    \textit{Package}: CSG324), carregar todos os arquivos-fonte do pipeline e realizar o
    mapeamento das suas portas de E/S --- isso é feito também pelo Xilinx ISE, da
    seguinte forma:
    \begin{itemize}
        \item Página \textit{Design} $\rightarrow$ \textit{User Constraints}
            $\rightarrow$ \textit{Create Timing Constraints};
        \item Se for importado o arquivo \texttt{Pipeline.ucf}, o mapeamento das portas
            de E/S já estará feito. Caso contrário, é necessário realizá-lo por meio de
            \textit{I/O Pin Planning} --- o programa Plan-Ahead será aberto;
        \item \textit{Implement Design};
        \item \textit{Generate Programming File}: vários arquivos serão gerados no
            diretório do projeto. O arquivo \texttt{Pipeline.bit} é o arquivo a ser
            carregado no FPGA.
    \end{itemize}
    Depois de gerado o arquivo \texttt{.bit} necessário, será utilizado o pacote
    Digilent Adept para carregá-lo na placa. Após conectar o FPGA no computador por
    um cabo USB, em um terminal do Linux, use os seguintes comandos:

    \texttt{djtgcfg enum} $\rightarrow$ Listar as placas reconhecidas pela porta USB;

    \texttt{djtgcfg init -d Nexys3} $\rightarrow$ Inicializar o dispositivo Nexys3;

    \texttt{djtgcfg prog -i 0 -d Nexys3 -f Pipeline.bit} $\rightarrow$ Carregar o arquivo
    \texttt{Pipeline.bit} no dispositivo Nexys3.

    Com o arquivo carregado, o FPGA começará a executar o programa.

\end{document}
