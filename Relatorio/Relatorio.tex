\documentclass[a4paper, 11pt]{article}
\usepackage[top=3cm, bottom=3cm, left = 2cm, right = 2cm]{geometry}
\usepackage[brazilian]{babel}
\usepackage{setspace}
\usepackage{graphicx}
\usepackage{multicol}

\title{Relatório: Trabalho Final - MIPS Pipeline}
\author{Luize Cunha Duarte - GRR20221232\\
    Gabriel Lisboa Conegero - GRR20221255\\
    Pedro Folloni Pesserl - GRR20220072\\
    Rebeca Soares de Oliveira - GRR20221260\\
\textit{Departamento de Informática}\\
\textit{Universidade Federal do Paraná - UFPR}\\
Curitiba, Brasil\\
\texttt{lcd22@inf.ufpr.br, glc22@inf.ufpr.br,}\\
\texttt{pfp22@inf.ufpr.br, rso22@inf.ufpr.br}}
\date{}

\begin{document}
\maketitle

\begin{abstract}
\begin{singlespace}
Este relatório documenta o processo de implementação do processador MIPS Pipeline de 16
bits para FPGA (Field Programmable Gate Array) através da linguagem VHDL (VHSIC Hardware
Description Language). O projeto do processador possui uma memória de dados e uma
memória de instruções de 128 KiB cada, e um banco de registradores com quatro
registradores. 
\end{singlespace}
\end{abstract}

\section{As Instruções.}
As divisões dos bits para os endereços e funcionalidades das instruções foi pensada
visando evitar bits soltos e manter um espaço considerável para cada funcionalidade.
Por isso, as instruções seguem a base do MIPS, tendo sua divisão reorganizada para
16 bits.
\begin{itemize}
    \item Tipo R (opcode: 3 bits; rs: 2 bits; rt: 2 bits; rd: 2 bits; shamt: 4 bits;
        func: 3 bits).
        \begin{figure}[h]
        \centering
        \includegraphics[width=.6\textwidth]{tipo-r.png}
        \caption{Formato das instruções do tipo R.}
        \end{figure}

    \item Tipo I. (opcode: 3 bits; rs: 2 bits; rt: 2 bits; imediato: 9 bits).
        \begin{figure}[h]
        \centering
        \includegraphics[width=.6\textwidth]{tipo-i.png}
        \caption{Formato das instruções do tipo I.}
        \end{figure}
\end{itemize}

As instruções suportadas são:
\begin{multicols}{3}
    \begin{itemize}
        \item ADD
        \item AND
        \item OR 
        \item NOR
        \item Set Less Than
    \end{itemize}
    \columnbreak
    \begin{itemize}
        \item Shift Left Logical
        \item Shift Right Logical
        \item ADDI
        \item ANDI
    \end{itemize}
    \columnbreak
    \begin{itemize}
        \item ORI
        \item Branch On Equal
        \item Load Word
        \item Store Word
    \end{itemize}
\raggedcolumns
\end{multicols}

\section{A arquitetura e sua implementação.}
Buscando uma arquitetura que suportasse todas as instruções decididas, sua organização
deu-se da maneira representada abaixo. Nesse sentido, cada componente foi implementado
utilizando a linguagem VHDL.

\begin{figure}[h!]
    \centering
    \includegraphics[width=.7\linewidth]{placeholder.jpg}
    \caption{Bloco operacional do MIPS Pipeline.}
\end{figure}

\section{Os componentes.}
\begin{enumerate}
    \item \textbf{Instruction Fetch}.

    Sendo a primeira parte da execução de uma instrução, é onde o ponteiro para a
    memória de instruções (Instruction Memory) é calculado e ela é lida, guardando tal
    instrução e sua localização no registrador intermediário IF\_ID.

    \item \textbf{Instruction Decode}.

    Após lida, a instrução será decodificada, gerando seus sinais de controle através
    da Control Unit, a qual recebe os três primeiros bits e retorna os sinais necessários
    ligados. Além disso, os registradores usados para realizar a operação desejada são
    lidos no Register Bank, que possui quatro registradores de 16 bits de capacidade,
    sendo o primeiro sempre 0. Por fim, tem-se também nesse estágio a Hazarding Unit,
    a qual reconhece se a instrução sendo executada na proxima etapa irá desviar o fluxo
    de instruções, zerando as instruções que não serão mais executadas, um processo
    denominado “flush”.

    \item \textbf{Execute}.

    Tendo os dados necessários para realizar a operação, nessa etapa temos a ULA,
    controlada pela ALUControl, a qual verifica se é uma instrução do tipo R para
    realizar o “func” – últimos três bits da instrução – ou então uma soma,
    subtração, AND ou OR.

    Nessa parte, também a calculado o endereço do Branch on Equal,  bastando concatenar
    “00” ao fim do immediato extendido (bits 13 ao 0) e somar ao novo PC, passado pelos
    registradores intermediarios. Por fim, um componente importante para o desempenho do
    pipeline, a Forwarding Unit, é implementado na etapa. Esse componente é responsável
    por evitar stalls no processador ao receber dados de etapas seguintes, não sendo
    necessária a espera da escrita nos registradores – detalhada mais a frente – das
    instruções anteriores.

    \item \textbf{Memory}.

    O único componente presente é a memória de dados (Memory Data), o componente mais
    lento do Pipeline. Usada apenas nas funções de Load Word e Store Word, a memória
    possui 128 KiB de armazenamento com 12 bits cada linha. Por fim, os dados carregados
    são passados para o registrador intermediário MEM\_WB.
    
    \item \textbf{Write-Back}.

    Sendo a última etapa do MIPS Pipeline, ela guarda no banco de registradores o
    dado de saída da ALU ou a palavra carregada da Memory Data.
\end{enumerate}

\end{document}
